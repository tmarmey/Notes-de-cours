\documentclass[12pt,a4paper]{article}
\usepackage[utf8]{inputenc}
\usepackage{graphicx}
\graphicspath{{../Images/}}
\usepackage{amsmath}
\usepackage{amsfonts}
\usepackage{amssymb}
\usepackage{hyperref}
\usepackage[margin=1in]{geometry}
\usepackage{subfig}
\usepackage{float}
\usepackage{xcolor}

\author{Thibaut Marmey}

\title{Notes de cours Deep-Learning}
\begin{document}
	\maketitle

\begin{normalsize}
\tableofcontents
\end{normalsize}

\section{Initiation à Jupyter Notebook}

\subsection{Importer des bibliothèques au kernel}
\begin{itemize}
\item \textit{import libraryName}
\item S'il y a un problème de librairie : \textit{pip install libraryName}
\begin{itemize}
\item \textit{matplotlib}
\item \textit{image}
\end{itemize}
\item \textit{os} : lire, écrire, ouvrir, manipuler des fichiers
\begin{itemize}
\item \textit{os.mkdir()}
\item \textit{os.listdir()}
\item \textit{[a for a in os.listdir('nameFolder') if '.jpg' in a or '.jpeg' in a]} (spécifier des fichiers)
\item var.endswith('.extension/smth') (renvoie True/False)
\item \textit{os.path.join(path,*paths)} : Join one or more path components intelligently. The return value is the concatenation of \textit{path} and any members of \textit{*paths}.
\end{itemize}
\item \textit{urlib.request} : intéragir avec les URLs
\item \textit{ssl} (Secure Sockets Layer) : créer une connection sécurisé (crypté) entre un client et un serveur.
\item \textit{matplotlib.pyplot as plt} : visualisation and chargement d'images
\newline On peut ainsi faire référence à \textit{matplotlib.pyplot} seulement avec \textit{plt}. C'est une pratique courante.
\begin{itemize}
\item \textit{\%matplotlib inline} : la visualisation se fera dans notebook et non pas avec une fenêtre pop-up (méthode par défaut)
\end{itemize}
\item \textit{numpy as np} : permet de travailler sur des variables numériques plus facilement (images par exemple)
\item 
\end{itemize}

\subsection{Généralité}
\subsubsection{RGB Image Representation}
\begin{itemize}
\item \textit{plt.imshow(img)}
\item \textit{img.shape} (lignes, colonnes, couleurs): dimension de la donnée (image)
\item \textit{img.dtype} : donne le nombre de bits utilisé pour coder l'image
\begin{itemize}
\item \textit{uint8 : unsigned, int, 8} : pas de signe (toutes les valeurs seront positives), seulement des entiers, codés sur 8 bits (de 0 à 255)
\end{itemize}
\end{itemize}

\section{Initiation au Deep-learning}
\subsection{Général}
\begin{itemize}
\item \href{https://github.com/pkmital/CADL}{Cours GitHub Kadenze}
\item \href{http://docs.continuum.io/anaconda/install/linux/}{Installation Anaconda}
\item Lancer anaconda : \textit{anaconda-navigator}
\item Créer un environnement virtuel via anaconda : \textit{conda create -n nomFile anaconda}
\item Installer TF dans cet env. : \textit{conda install -c conda-forge tensorflow}
\item \textit{conda install jupyter notebook}
\item Lancer env. virtuel : \textit{source activate tensorflow} (\textit{tensorflow} est le nom donné à l'env. virtuel)
\item Lancer Jupyter : \textit{jupyter notebook}
\item Tester Tensorflow : \textit{python -c "import tensorflow as tf; print(tf.\textunderscore \textunderscore version\textunderscore \textunderscore)"}
\item Quitter l'env. virtuel : \textit{source deactivate}
\end{itemize}

\subsection{CADL session}
\subsubsection{Session-0}
\begin{itemize}
\item Dataset \href{http://mmlab.ie.cuhk.edu.hk/projects/CelebA.html}{\textit{Celeb Net}}, à peu près 200 000 images de célébrités. 
\end{itemize}


\end{document}