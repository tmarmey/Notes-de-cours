\documentclass[12pt,a4paper]{article}
\usepackage[utf8]{inputenc}
\usepackage{graphicx}
\graphicspath{{../Images/}}
\usepackage{amsmath}
\usepackage{amsfonts}
\usepackage{amssymb}
\usepackage{hyperref}
\usepackage[margin=1in]{geometry}
\usepackage{subfig}
\usepackage{float}
\usepackage{xcolor}
\usepackage{listings}
\definecolor{dkgreen}{rgb}{0,0.6,0}
\definecolor{gray}{rgb}{0.5,0.5,0.5}
\definecolor{mauve}{rgb}{0.58,0,0.82}

\lstset{frame=tb,
  backgroundcolor=\color[rgb]{0.93,0.93,0.93},
  language=Python,
  aboveskip=3mm,
  belowskip=3mm,
  showstringspaces=false,
  columns=flexible,
  basicstyle={\small\ttfamily},
  numbers=none,
  numberstyle=\tiny\color{gray},
  keywordstyle=\color{blue},
  commentstyle=\color{dkgreen},
  stringstyle=\color{mauve},
  breaklines=true,
  breakatwhitespace=true,
  tabsize=3
}

\author{Thibaut Marmey}

\title{Notes de cours CADL - session-2}

\begin{document}
	\maketitle

\begin{scriptsize} \begin{itemize}
\item The basic components of a neural network
\item How to use gradient descent to optimize parameters of a neural network
\item How to create a neural network for performing regression
\end{itemize}\end{scriptsize}

\begin{normalsize}
\tableofcontents
\end{normalsize}

\section{Introduction}
\subsection{Generalities}
\begin{itemize}
\item Use data and gradient descent to teach the network what the values of the parameters of the network should be.
\item Idea of machine learning : letting the machine learn from the data.
\item We are interested in letting the computer figure out what representations it needs in order to better describe the data and some objective we've defined.
\end{itemize}
\end{document}