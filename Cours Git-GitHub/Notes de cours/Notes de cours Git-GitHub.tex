\documentclass[12pt,a4paper]{article}
\usepackage[utf8]{inputenc}
\usepackage{graphicx}
\graphicspath{{../Images/}}
\usepackage{amsmath}
\usepackage{amsfonts}
\usepackage{amssymb}
\usepackage{hyperref}
\usepackage[margin=1in]{geometry}
\usepackage{subfig}
\usepackage{float}

\author{Thibaut Marmey}

\title{Notes de cours Git/GitHub}
\begin{document}
	\maketitle

\begin{normalsize}
\tableofcontents
\end{normalsize}

\section{Git}
\subsection{Les commandes de bases}
\begin{itemize}
\item Activer un repository Git, se placer dans le dossier que l’on veut et faire : 
\newline \textit{git init }
\item L’activation du repository Git génère un index. Lorsqu’on rajoute un nouveau fichier dans le repository il faut rajouter ce fichier à l’index par la commande :
\newline \textit{git add nomDuFichier.extension}
\item Ajouter tous les fichiers du dossier courant :
\newline \textit{git add .}
\item Enregistrement des modifications dans le rep :
\newline \textit{git commit -m "ajout de fichier"} (le -m permet de rajouter un message au commit)
\item Pour se positionner sur un commit de l'historique :
\newline \textit{git checkout SHADuCommit}
\item Revenir au commit le plus récent : 
\newline \textit{git checkout master}
\item Créer un nouveau commit qui fait l'inverse du commit précédent :
\newline \textit{git revert SHADuCommit}
\item Modifier le message du dernier commit
\newline \textit{git commit --amend -m "votre message"}
\item Annuler tous les changements qui ne sont pas encore commités :
\newline \textit{git reset --hard}
\item Pour créer une nouvelle branche, se placer au commit avec un checkout et faire :
\newline \textit{git checkout -b nouvelle-branche}
\item Pour supprimer une branche, se placer à l'extérieur de celle-ci et fait :
\textit{git branch -d nom-de-la-branche}
\item Copier un repo
\end{itemize}

\end{document}