\documentclass[12pt, letterpaper]{article}
\usepackage[utf8]{inputenc}
\usepackage{graphicx}
\graphicspath{{../Images/}}
\usepackage{hyperref}
\usepackage{xcolor}
\hypersetup{
    linktoc=all,     %set to all if you want both sections and subsections linked
    linkcolor=blue,  %choose some color if you want links to stand out
}
\usepackage[margin=1in]{geometry}
\usepackage{subfig}
\usepackage{float}
\usepackage{ulem}
\title{Notes de cours : \LaTeX}
\author{Thibaut Marmey}
\date{Août 2018}

\begin{document}

\maketitle
% Afficher une image centrée
\begin{center}
\includegraphics[scale=0.3]{CV_photo_Thibaut_Marmey}
%\begin{figure}
%\caption{image photo de CV}
%\end{figure}
\end{center}
\vspace{0.25cm}
% Modifier le nom de base de la section "Abstract" en "Résumé"
\renewcommand{\abstractname}{Résumé}
\begin{abstract}
Ce document présente les commandes utiles pour créer un document \LaTeX.
\end{abstract}

\tableofcontents

\newpage
\section{Some commands to start a document}
\begin{enumerate}
\item Préambule : type de document
\begin{enumerate}
\item $\backslash$documentclass[ ]$\lbrace$ $\rbrace$ : \textit{twocolumn, twoside}
\item $\backslash$usepackage[utf8]{inputenc}
\item $\backslash$title$\lbrace$ $\rbrace$
\item $\backslash$author$\lbrace$ $\rbrace$
\item $\backslash$date$\lbrace$ $\rbrace$
\end{enumerate}
\end{enumerate}

\section{Display}
\begin{enumerate}
\item $\backslash$begin$\lbrace$document$\rbrace$  $\backslash$end$\lbrace$document$\rbrace$
\item $\backslash$begin$\lbrace$titlepage$\rbrace$  $\backslash$end$\lbrace$titlepage$\rbrace$ \textit{environnement pour la première page}
\item $\backslash$maketitle : \textit{écrit les éléments title, author, dates, etc...}
\item Pour afficher une table des matières cliquable
\begin{itemize}
\item \textbackslash usepackage\{hyperref\}
\item \textbackslash hypersetup\{linktoc=all, linkcolor=blue\}
\item \textbackslash usepackage[margin=1in]\{geometry\} : \textit{changer les marges de la page}
\end{itemize}
\end{enumerate}

\section{Basic formatting}
\begin{enumerate}
\item $\backslash$begin$\lbrace$abstract$\rbrace$  $\backslash$end$\lbrace$abstract$\rbrace$ : \textit{introduction du document}
\item $\backslash$newline ou $\backslash$ $\backslash$ : \textit{permet de sauter une ligne}
\item \textit{Pour commencer un autre paragraphe taper deux fois sur la touche "Entrée"}
\item \% ou \textit{ctrl + t} : \textit{créer un commentaire}
\item \textbackslash begin\{enumerate\} \textbackslash end\{enumerate\} : \textit{créer une liste énumérée}
\item \textbackslash item
\item \textbackslash begin\{center\} \textbackslash end\{center\}
\item structure
\begin{enumerate}
\item Partie : \textbackslash part\{nom\}
\item Chapitre : \textbackslash chapter\{nom\}
\item Section : \textbackslash section\{nom\}
\item Sous-section : \textbackslash subsection\{nom\}
\item Sous-sous-section : \textbackslash subsubsection\{nom\}
\item Paragraphe : \textbackslash paragraph\{nom\}
\item Sous-paragraph : \textbackslash subparagraph\{nom\}
\end{enumerate}
\item Texte barré : \textbackslash usepackage\{ulem\}, \textbackslash sout\{\sout{texte barré}\}
\item \textbackslash pagebreak ou \textbackslash newpage \textit{permet de commencer une nouvelle page}
\item \textbackslash usepackage\{hypereff\} \textbackslash href\{url\}\{description\} : \textit{\href{https://en.wikibooks.org/wiki/LaTeX/Hyperlinks}{exemple}}
\newline Il est possible aussi d'accéder à un dossier ou un fichier via \textit{href} :
\newline \textbackslash href\{run:pathFolder ou run:pathfile/file.pdf\}
\item \textbackslash begin\{description\} \textbackslash item [nom] : texte \textbackslash end\{description\} : \textit{pour pouvoir faire comme ci-dessous}
\begin{description}
\item [nom] : texte
\end{description}
\item \textbackslash vspace\{size in ou cm\}; \textbackslash hspace\{size in ou cm\}
\item Créer un tableau. \textbackslash begin\{tabular\} vertical separation \textbackslash hline
\newline \begin{tabular}{|l|c|r|}
  \hline
  colonne 1 & colonne 2 & colonne 3 \\
  \hline
  1.1 & 1.2 & 1.3 \\ \hline
  2.1 & 2.2 & 2.3 \\
  \hline
\end{tabular}
\item Ajouter des couleurs au texte : \textbackslash usepackage\{xcolor\}
\newline \textcolor{red}{\textbackslash textcolor\{red\}\{text\}}
\newline \colorbox{blue!30}{\textbackslash colorbox\{blue!30\}}
\newline \colorbox{green}{\textbackslash colorbox\{green\}}
\item Pour écrire les symbole \textless et \textgreater écrire respectivement \textbackslash textless et \textbackslash textgreater
\item Écrire un signe supérieur ou inférieur avec \$\textbackslash langle\$ $\langle$ ou \$\textbackslash rangle\$ $\rangle$
\end{enumerate}

\section{Basic Image management}
\begin{enumerate}
\item Gérer les images
\item \textbackslash usepackage\{graphicx\}, \textbackslash graphicspath\{\{path\}\} et \textbackslash includegraphics\{nom de l'image\}
\item Afficher deux images une à coté de l'autre : utiliser les packages \textbackslash usepackage\{subfig\} et \textbackslash usepackage\{float\}
\newline Créer un environnement \textit{minipage} et un environnement \textit{figure}. Utiliser ensuite les fonction \textit{subfloat} et \textit{includegraphics}. Entre les images utiliser \textit{\textbackslash hfill} ou \textit{\textbackslash hspace\{1cm\}}.
\end{enumerate}
\begin{minipage}{\linewidth}
  \begin{figure}[H]
  \centering
    \subfloat[Logo \LaTeX ]{\includegraphics[scale=0.4]{logoLatex}}\hspace{1cm}
    \subfloat[Logo Texmaker]{\includegraphics[scale=0.4]{logoTexmaker}}
  \end{figure}
\end{minipage}

\section{Raccourcis clavier}
\begin{enumerate}
\item Pour rechercher/remplacer : \textit{ctrl + R}
\end{enumerate}

\end{document}
