\documentclass[12pt,a4paper]{article}
\usepackage[utf8]{inputenc}
\usepackage{graphicx}
\graphicspath{{../Images/}}
\usepackage{amsmath}
\usepackage{amsfonts}
\usepackage{amssymb}
\usepackage{hyperref}
\usepackage[margin=1in]{geometry}
\usepackage{subfig}
\usepackage{float}

\author{Thibaut Marmey}

\title{Notes de cours Python}
\begin{document}
	\maketitle

\begin{normalsize}
\tableofcontents
\end{normalsize}

\section{Commandes générales}
\subsection{Environnement}
\begin{itemize}
\item \href{https://docs.python.org/3/tutorial/index.html}{Tutorial 1 python}
\item \href{https://docs.python.org/fr/3/tutorial/controlflow.html}{Suite du tutorial : page où on s'est arrêté}
\item \href{run:../Test Python/}{Programmes python}
\item Changer les permissions du programme .py : \textit{chmod -x nom.py}
\item Exécuter le programme : \textit{python nom.py}
\end{itemize}

\subsection{Le langage}
\subsubsection{Infos}
\begin{itemize}
\item La division renvoie toujours un \textit{double}
\item Mettre un nombre au carré : utiliser \textit{**}
\item Attendre une entrée clavier : \textit{x = int(input("Please enter an integer: "))}
\item Pour commenter en multi lignes, il faut utiliser trois guillements : 
\newline \textit{""" commentaires """}

\subsubsection{Instructions utiles}
\item Ajouter élément à la fin de la liste : \textit{liste.append[value]}
\item Insérer élément dans une liste : \textit{liste.insert(indice, value)}
\item Itérer sur une suite de nombre : \textit{range(nombre)} (utiliser dans boucle \textit{for})
\newline C'est un objet dit \textit{itérable} c'est à dire qu'elle est utilisée par des \textit{itérateurs} comme \textit{for} ou \textit{list()}. Pour obtenir une liste de nombre successif on peut ainsi combiner \textit{range()} et \textit{list()} : \textit{list(rang(n))}
\item L'instruction \textit{pass} ne fait rien, elle permet de de fournir une syntaxe correcte comme : \textit{while True: pass}

\subsubsection{Les string}
\item Dans un \textit{print()}, si l'on ne veut pas que les charactères qui précèdent le charactère \textit{\textbackslash} on peut utiliser le paramètre \textit{row : r}.
\newline \textit{print(r"mot\textbackslash nom")} : ainsi la combinaison \textit{\textbackslash n} n'est pas considérée comme un saut de ligne
\item Pour modifier le comportement de saut de lige par défaut de \textit{print}. Rajouter la valeur \textit{end} : \textit{print(a, end = ', ')}
\item Opération sur les strings : possibilité des les additionner entre eux, de les multiplier par un entier : \textit{3 * 'un' + 'deux'}
\item Accéder aux élements d'un tableau ou string etc... : \textit{tab[indice]}. L'indice commence à 0. On peut accéder au dernier élément par l'indice -1.
\item Les strings sont immuables c'est à dire qu'on ne peut pas modifier d'élément. On peut seulement lire les éléments.
\item Récupérer plusieurs éléments du tableau : \textit{tableau[n:m]}. Permet de récupérer l'élément n juqu'a m-1.
\item Récupérer taille d'un tableau : \textit{len(tableau)}

\subsubsection{Les listes}
\item Déclaration : \textit{liste = [1, 2 ,3]}
\item Concaténation : \textit{liste1 + liste2}
\item Les listes sont muables (contrairement aux strings) : \textit{liste[1] = 20}
\item Remplacer des valeurs : \textit{liste[0:3]=[10,20,30]}
\item Effacer la liste ou certains éléments : \textit{liste[:] = []}
\item Imbriquer des listres entre elles : \textit{liste3 = [liste1, liste2]}.  Les listes n'ont pas besoin de posséder le même type de valeur.
\newline Accéder au premier élément de liste2 :\textit{liste3[1][0]}
\item Affectation multiples de variables : \textit{a, b = b, a+b}. Cepdant lors du calcul de \textit{b}, \textit{a} a toujours la même valeur et n'est pas encore égale à \textit{b}.


\subsubsection{Les boucles}
\item Les boucles \textit{for, while} finissent par un \textit{" : "} et ne nécessitent pas de parenthèse.
\item A l'intérieur de ces fonctions, l'indentation est primordial puisque c'est la méthode utilisée par python pour regrouper les instructions. 
\item Pour signaler la fin d'une fonction un double saut de ligne est nécessaire.
\item On peut combiner les boucles \textit{while} et \textit{for} avec des \textit{break}, \textit{continue} et \textit{else}. 
\newline En combinant un \textit{for avec un break à la fin} suivi d'un \textit{else}. Ainsi on va d'abord aller dans la boucle \textit{for}, si une condition est bonne on break la boucle \textit{for} et donc on ne va pas dans le \textit{else}. Si au contraire à la fin de la boucle \textit{for} on n'a pas eu de \textit{break} on va alors passer dans la boucle \textit{else}.
\item \textit{Continue} quant à lui fait passer directement la boucle à l'instruction suivante.

\subsubsection{Boucle \textit{for}}
\item L'instruction for itère sur les éléments d'une séquence : \textit{for w in words:}
\item Si l'on veut modifier la séquence sur laquelle on itère, il faut créer une copie de cette séquence dans l'instruction du \textit{for}: \textit{for w in ranges words\textbf{[:]}}
\item Itérer sur une suite de nombre : \textit{for i in range(5)} : génère 5 valeurs de 0 à 5 exclu (le dernier élement n'est jamais pris en compte)

\subsubsection{Condition \textit{if}}
\item Finir l'instruction par un deux-points \textit{:}
\item Les autres instructions sont : \textit{elseif cd1:} et \textit{else:}
\item Toujours sauter des lignes entre ces instructions et faire attention à l'indentation

\subsubsection{Les fonctions}
\item Définir une fonction par \textit{def}
\item Faire attention à l'indentation
\item Chaîne de documentation : explication succinte de la fonction dans les premières lignes de la fonction.

\end{itemize}


\end{document}