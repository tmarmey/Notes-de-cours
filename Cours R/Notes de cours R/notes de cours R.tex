\documentclass[12pt,a4paper]{article}
\usepackage[utf8]{inputenc}
\usepackage{graphicx}
\graphicspath{{../Images/}}
\usepackage{amsmath}
\usepackage{amsfonts}
\usepackage{amssymb}
\usepackage{hyperref}
\usepackage[margin=1in]{geometry}
\usepackage{subfig}
\usepackage{float}

\author{Thibaut Marmey}

\title{Notes de cours R}
\begin{document}
	\maketitle

\begin{normalsize}
\tableofcontents
\end{normalsize}

\section{Programmation R}
\subsection{Généralité}
\begin{itemize}
\item \href{https://linuxconfig.org/rstudio-on-ubuntu-18-04-bionic-beaver-linux}{\textit{lien internet : }doc d'installation de R et Rstudio}
\item Utilisation de la documentation "aide" : \textit{help()} ou \textit{help("nom de la fonction")} ou  \textit{?log}
\item Documentation plus détaillée sur internet : \textit{help.start()}
\item Interaction avec l'environnement R :
\begin{itemize}
\item liste toutes les variables crées : \textit{ls()}
\item liste des variables avec une succession de lettre particulière : \textit{ls(pattern="mot")}
\item supprimer des variables : \textit{rm(var)}\\
\item quitter le travail : \textit{q()} ou \textit{quit()}
\end{itemize}
\item Ecrire des commentaires avec "\#"
\item Utiliser la fonction \textit{print()} pour afficher les informations, textes sur la console
\item Enregistrer les variables que l'on veut : \textit{save(var1, var2, ..., file = "nameFile.extension")}.
\newline Enregistrement de ces fichiers dans le répertoires associés \textit{Tools/Global Options}, ici c'est Documents/Notes-de-cours/Cours R/Test
\item 
\end{itemize}

\subsection{Fonctions}
\begin{itemize}
\item Tester le type d'une variable :
\begin{itemize}
\item \textit{is.character(var)}
\item \textit{is.numeric(var)}
\item \textit{is.logical(var)}
\end{itemize}
\item Spécifier le typage de la variable :
\begin{itemize}
\item \textit{as.numeric(var)}
\item \textit{as.logical(var)}
\item \textit{as.character(var)}
\end{itemize}
\item Renvoyer l'entier inférieur : \textit{floor(nb)}
\item Renvoyer l'entier supérieur : \textit{ceiling(nb)}
\item Arrondir à l'entier le plus proche : \textit{round(nb)}
\item fonction trigo : \textit{cos(angleRad), sin, ...}
\end{itemize}

\subsection{Manipulation de chaînes de caractères}
\begin{itemize}
\item Saisir des données depuis le clavier : \textit{scan()}. Pour valider l'input appuyer sur \textit{entrée}, et un nouvel input appararaîtra. Pour arrêter la prise d'input appuyer directement sur \textit{entrée}.
\item Rentrer le nombre d'input avec l'appel de \textit{scan(nmax=n)}
\item La concaténation de texte et nombre avec \textit{paste(texte, nombres, ...)}
\item Combinaison de \textit{scan()} et \textit{paste()} : \textit{paste('Bonjour j'ai ', scan(nmax=1), ' ans')}
\item Permet de compter le nombre de lettres présentes dans la chaîne de caractères : \textit{nchar("chaîne")}
\item Mettre les caractères en majuscules ou minuscules : \textit{toupper(), tolower()}
\item Extraire une sous chaîne de caractères avec point de départ et d'arrivée inclus : \textit{substr("chaine", start, stop)}. La chaîne de caractères commence à 1.
\end{itemize}

\end{document}